\section*{Gniazda w języku Python - moduł socket} \mbox{}\\

\noindent \textbf{Tworzenie gniazd klienckich/serwerowych TCP, IPv4}

\begin{code}
#!/usr/bin/env python3
import socket

if __name__ == '__main__':

	sockIPv4 = socket.socket(socket.AF_INET, socket.SOCK_STREAM)
	sockIPv4.close()
\end{code} \mbox{}\\

\noindent \textbf{Tworzenie gniazd klienckich/serwerowych TCP, IPv6}

\begin{code}
#!/usr/bin/env python3
import socket

if __name__ == '__main__':

	sockIPv6 = socket.socket(socket.AF_INET6, socket.SOCK_STREAM)
	sockIPv6.close()
\end{code} \mbox{}\\

\noindent \textbf{Tworzenie gniazd klienckich/serwerowych UDP, IPv4}

\begin{code}
#!/usr/bin/env python3
import socket 

if __name__ == '__main__':

	sockIPv4 = socket.socket(socket.AF_INET,  socket.SOCK_DGRAM)
	sockIPv4.close()
\end{code}\mbox{}\\

\noindent \textbf{Tworzenie gniazd klienckich/serwerowych UDP, IPv6}

\begin{code}
#!/usr/bin/env python3
import socket 

if __name__ == '__main__':

	sockIPv6 = socket.socket(socket.AF_INET6,  socket.SOCK_DGRAM)
	sockIPv6.close()
\end{code}\mbox{}\\

\newpage
\noindent \textbf{Gniazdo klienckie TCP, IPv4: nawiązanie połączenia z serwerem}

\begin{code}
#!/usr/bin/env python3
import socket

if __name__ == "__main__":

    sockIPv4 = socket.socket(socket.AF_INET, socket.SOCK_STREAM)
    sockIPv4.settimeout(5)

    try:
        sockIPv4.connect(("127.0.0.1", 80))
    except socket.error as exc:
        print(f"Wyjatek socket.error: {exc}")

    sockIPv4.close()
\end{code}\mbox{}\\


\noindent \textbf{Gniazdo klienckie TCP, IPv6: nawiązanie połączenia z serwerem}

\begin{code}
#!/usr/bin/env python3
import socket

if __name__ == "__main__":

    address = socket.getaddrinfo("::1", 80, socket.AF_INET6)
    sockIPv6 = socket.socket(socket.AF_INET6, socket.SOCK_STREAM)
    sockIPv6.settimeout(5)

    try:
        sockIPv6.connect(address[0][4])
    except socket.error as exc:
        print(f"Wyjatek socket.error: {exc}")

    sockIPv6.close()
\end{code}\mbox{}\\

\newpage
\noindent \textbf{Gniazdo klienckie TCP, IPv4: komunikacja z serwerem (wysyłanie i odbieranie danych)}

\begin{code}
#!/usr/bin/env python3
import socket

if __name__ == "__main__":

    sockIPv4 = socket.socket(socket.AF_INET, socket.SOCK_STREAM)
    sockIPv4.settimeout(5)

    try:
        sockIPv4.connect(("127.0.0.1", 80))
        sockIPv4.sendall("Hello Server!".encode())	# wysylanie
        print(sockIPv4.recv(1024).decode())								# odbieranie
    except socket.error as exc:
        print(f"Wyjatek socket.error: {exc}")

    sockIPv4.close()
\end{code}\mbox{}\\

\noindent \textbf{Gniazdo klienckie TCP, IPv6: komunikacja z serwerem (wysyłanie i odbieranie danych)}

\begin{code}
#!/usr/bin/env python3
import socket

if __name__ == "__main__":

    address = socket.getaddrinfo("::1", 80, socket.AF_INET6)
    sockIPv6 = socket.socket(socket.AF_INET6, socket.SOCK_STREAM)
    sockIPv6.settimeout(5)

    try:
        sockIPv6.connect(address[0][4])
        sockIPv6.sendall("Hello Server!".encode())	# wysylanie
        print(sockIPv6.recv(1024).decode())								# odbieranie
    except socket.error as exc:
        print(f"Wyjatek socket.error: {exc}")

    sockIPv6.close()
\end{code}\mbox{}\\


