\section*{Gniazda w języku Python - moduł socket} \mbox{}\\

\noindent \textbf{Tworzenie gniazd serwerowych TCP, IPv4}

\begin{code}
#!/usr/bin/env python3
import socket

if __name__ == '__main__':

	sockIPv4 = socket.socket(socket.AF_INET, socket.SOCK_STREAM)
	sockIPv4.setsockopt(socket.SOL_SOCKET, socket.SO_REUSEADDR, 1)
	sockIPv4.close()
\end{code} \mbox{}\\

\noindent \textbf{Tworzenie gniazd serwerowych TCP, IPv6}

\begin{code}
#!/usr/bin/env python3
import socket

if __name__ == '__main__':

	sockIPv6 = socket.socket(socket.AF_INET6, socket.SOCK_STREAM)
	sockIPv6.setsockopt(socket.SOL_SOCKET, socket.SO_REUSEADDR, 1)
	sockIPv6.close()
\end{code} \mbox{}\\

\noindent \textbf{Tworzenie gniazd serwerowych UDP, IPv4}

\begin{code}
#!/usr/bin/env python3
import socket 

if __name__ == '__main__':

	sockIPv4 = socket.socket(socket.AF_INET,  socket.SOCK_DGRAM)
	sockIPv4.setsockopt(socket.SOL_SOCKET, socket.SO_REUSEADDR, 1)
	sockIPv4.close()
\end{code}\mbox{}\\

\noindent \textbf{Tworzenie gniazd serwerowych UDP, IPv6}

\begin{code}
#!/usr/bin/env python3
import socket 

if __name__ == '__main__':

	sockIPv6 = socket.socket(socket.AF_INET6,  socket.SOCK_DGRAM)
	sockIPv6.setsockopt(socket.SOL_SOCKET, socket.SO_REUSEADDR, 1)
	sockIPv6.close()
\end{code}\mbox{}\\

\newpage
\noindent \textbf{Gniazdo serwerowe TCP, IPv4: nasłuchiwanie na danym porcie}

\begin{code}
#!/usr/bin/env python3
import socket

if __name__ == "__main__":

	sockIPv4 = socket.socket(socket.AF_INET, socket.SOCK_STREAM)
	sockIPv4.setsockopt(socket.SOL_SOCKET, socket.SO_REUSEADDR, 1)

	server_address = ('127.0.0.1', 80)
	sockIPv4.bind(server_address)
	sockIPv4.listen(1)

	sockIPv4.close()
\end{code}\mbox{}\\


\noindent \textbf{Gniazdo serwerowe TCP, IPv6: nasłuchiwanie na danym porcie}

\begin{code}
#!/usr/bin/env python3
import socket

if __name__ == "__main__":


    sockIPv6 = socket.socket(socket.AF_INET6, socket.SOCK_STREAM)
    sockIPv6.setsockopt(socket.SOL_SOCKET, socket.SO_REUSEADDR, 1)

    sockIPv6.bind(("::1", 80))
    sockIPv6.listen(1)
    
    sockIPv6.close()
\end{code}\mbox{}\\

\newpage
\noindent \textbf{Gniazdo serwerowe TCP, IPv4: komunikacja z klientem (wysyłanie i odbieranie danych)}

\begin{code}
#!/usr/bin/env python3
import socket

if __name__ == "__main__":

    sockIPv4 = socket.socket(socket.AF_INET, socket.SOCK_STREAM)
    sockIPv4.setsockopt(socket.SOL_SOCKET, socket.SO_REUSEADDR, 1)

    server_address = ("127.0.0.1", 80)
    sockIPv4.bind(server_address)
    sockIPv4.listen(1)

    connection, client_address = sockIPv4.accept()

    try:
        data = connection.recv(1024).decode()
        if data:
            connection.sendall(data.encode())
            print(data)
    finally:
        connection.close()
\end{code}\mbox{}\\

\noindent \textbf{Gniazdo serwerowe TCP, IPv6: komunikacja z klientem (wysyłanie i odbieranie danych)}

\begin{code}
#!/usr/bin/env python3
import socket

HOST = '::1'  # localhost
PORT = 80

def main():
    with socket.socket(socket.AF_INET6, socket.SOCK_STREAM) as server_socket:
        server_socket.bind((HOST, PORT))
        server_socket.listen(1)
        print(f'Server listening on {HOST}:{PORT}')

        client_socket, client_addr = server_socket.accept()
        print(f'Connected to client {client_addr[0]}:{client_addr[1]}')

        data = client_socket.recv(1024)					# odbieranie
        if data:
            print(f'Received data: {data.decode()}')
            client_socket.sendall(data)					# wysylanie

        client_socket.close()
        print(f'Connection closed with client {client_addr[0]}:{client_addr[1]}')

if __name__ == '__main__':
    main()
\end{code}\mbox{}\\

\newpage
\noindent \textbf{Gniazdo serwerowe UDP, IPv4: komunikacja z klientem (wysyłanie i odbieranie danych)}

\begin{code}
#!/usr/bin/env python3
import socket

if __name__ == "__main__":
    host = "127.0.0.1"   
    port = 80   
    
    server_socket = socket.socket(socket.AF_INET, socket.SOCK_DGRAM)
    server_socket.bind((host, port))

    print(f"UDP Echo Server is listening on {host}:{port}...")

    data, address = server_socket.recvfrom(1024)					# odbieranie
    print(f"Received data from {address[0]}:{address[1]}: {data.decode()}")

    server_socket.sendto(data, address)														# wysylanie
    print(f"Sent data back to {address[0]}:{address[1]}: {data.decode()}")

    server_socket.close()
\end{code}\mbox{}\\

\noindent \textbf{Gniazdo serwerowe UDP, IPv6: komunikacja z klientem (wysyłanie i odbieranie danych)}

\begin{code}
#!/usr/bin/env python3

import socket

HOST = "::1"
PORT = 80

sockIPv6 = socket.socket(socket.AF_INET6, socket.SOCK_DGRAM)
sockIPv6.close()
\end{code}

