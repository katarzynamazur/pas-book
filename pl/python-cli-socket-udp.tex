\subsubsection*{Gniazda w języku Python - moduł socket} \mbox{}\\


\noindent \textbf{Tworzenie gniazd klienckich/serwerowych UDP, IPv4}

\begin{code}
#!/usr/bin/env python3
import socket 

if __name__ == '__main__':
	sockIPv4 = socket.socket(socket.AF_INET,  socket.SOCK_DGRAM)
	sockIPv4.close()
\end{code} 

\noindent \textbf{Tworzenie gniazd klienckich/serwerowych UDP, IPv6}

\begin{code}
#!/usr/bin/env python3
import socket 

if __name__ == '__main__':
	sockIPv6 = socket.socket(socket.AF_INET6,  socket.SOCK_DGRAM)
	sockIPv6.close()
\end{code} 


\noindent \textbf{Gniazdo klienckie UDP, IPv4: komunikacja z serwerem (wysyłanie i odbieranie danych)}

\begin{code}
#!/usr/bin/env python3

import socket

HOST = '127.0.0.1'
PORT = 80

sockIPv4 = socket.socket(socket.AF_INET, socket.SOCK_DGRAM)
server_address = (HOST, PORT)

try:
    message = "Hello Server!"
    sent = sockIPv4.sendto(message.encode(), server_address)		# wysylanie
    data, server = sockIPv4.recvfrom(4096)																				# odbieranie
    print(f"Received: {data.decode()}")
finally:
    sockIPv4.close()
\end{code} 
 
\noindent \textbf{Gniazdo klienckie UDP, IPv6: komunikacja z serwerem (wysyłanie i odbieranie danych)}

\begin{code}
#!/usr/bin/env python3

import socket

HOST = "::1"
PORT = 80

sockIPv6 = socket.socket(socket.AF_INET6, socket.SOCK_DGRAM)

try:
    sockIPv6.sendto("Hello Server!".encode(), (HOST, PORT))			# wysylanie
    data, server = sockIPv6.recvfrom(4096)																				# odbieranie
    print(f"Received: {data.decode()}")
finally:
    sockIPv6.close()
\end{code}

