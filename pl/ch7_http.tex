%\noindent \underline{\textbf{Uwaga:}} W niniejszym rozdziale omawiamy protokół HTTP w wersji 1.1. \\

\noindent \textbf{Protokół HTTP (\href{http://www.jmarshall.com/easy/http/}{\textit{Hypertext Transfer Protocol}}, RFC 2616 oraz od 7230 do 7235)} - protokół warstwy aplikacji, wykorzystujący na niższej warstwie (zazwyczaj) gniazda TCP/IP oraz 2 domyślne porty: port niezabezpieczony \texttt{80} i port zabezpieczony: \texttt{443}.\\


\noindent Podstawowe informacje:

\begin{itemize}
\item Protokół HTTP jest protokołem wykorzystywanym do przesyłania plików (ogólnie mówiąc: zasobów) w sieci WWW (World Wide Web), bez względu na to, czy zasobem jest plik HTML, plik graficzny, wynik zapytania, czy cokolwiek innego
\item Protokół HTTP, do wersji \texttt{1.1}, jest protokołem tekstowym, gdzie komendy protokołu, podobnie jak w SMTP, POP3 czy IMAP są komendami tekstowymi, zrozumiałymi dla człowieka
\item HTTP to protokół typu zapytanie-odpowiedź. Zapytanie, wysyłane przez klienta, zawiera informację o żądanym zasobie. Odpowiedź, wysyłana przez serwer, zawiera treść zasobu. Jeśli serwer nie jest w stanie zwrócić odpytywanego zasobu, odpowiedź zawiera kod reprezentujący powód, dla którego zasób nie mógł być wysłany (np. zasób nie istnieje)
\item Formaty zapytania i odpowiedzi HTTP są do siebie podobne; zarówno zapytanie, jak i odpowiedź HTTP zawierają (linia początkowa i nagłówki powinny się kończyć parą znaków \texttt{CRLF},  czyli \texttt{\textbackslash r\textbackslash n}):

\begin{itemize}
\item linię początkową 
\item 0 lub więcej nagłówków
\item pustą linię (\texttt{CRLF},  czyli \texttt{\textbackslash r\textbackslash n})
\item opcjonalne ciało wiadomości
\end{itemize}

\textbf{Przykład}:\\

\begin{code}
linia poczatkowa, inna dla zadania, inna dla odpowiedzi \r\n
naglowek1: wartosc1 \r\n
naglowek2: wartosc2 \r\n
naglowek3: wartosc3 \r\n
\r\n
cialo wiadomosci, moze sie skladac z 1 lub
wielu linii, lub moze byc puste
\end{code}

\item Nagłówki HTTP to wszelkie komendy używane do komunikacji między przeglądarką WWW (klientem) a serwerem. Nagłówki są to właściwości żądania i odpowiedzi przesyłane wraz z samą wiadomością. Służą one przede wszystkim do sterowania zachowaniem serwera oraz przeglądarki przez nadawcę wiadomości.

\item Jeśli klient wysyła żądanie do serwera HTTP, żądanie powinno zawsze być zakończone parą znaków \texttt{CRLF}   (czyli \texttt{\textbackslash r\textbackslash n})
\item Serwer odsyłając odpowiedź HTTP nie określa za pomocą żadnych specjalnych znaków końca odsyłanej odpowiedzi. W przypadku, gdy chcemy mieć pewność, że odebraliśmy całą odpowiedź serwera HTTP, musimy parsować odebrane nagłówki  (\texttt{Content-Length} lub \texttt{Transfer-Encoding}), w których może znajdować się informacja o tym, jaki jest rozmiar odpowiedzi serwera, i użyć tej inforamcji do odebrania całej wiadomości. W przypadku, gdy serwer w odpowiedzi HTTP nie odeśle żadnego z powyższych nagłówków, aby mieć pewność odebrania całej odpowiedzi od serwera, musimy odbierać dane, dopóki serwer nie zakończy / zamknie połącznia. Zgodnie z formatem żądania i odpowiedzi HTTP, nagłówki od ciała oddzielają znaki \texttt{CRLF} \texttt{CRLF} (czyli \texttt{\textbackslash r\textbackslash n \textbackslash r\textbackslash n}).

\end{itemize}

\newpage 
\noindent {\underline{Żądania HTTP}} \\

\begin{itemize}
\renewcommand{\labelitemi}{\scriptsize}

\item Ogólny format \href{https://www.w3.org/Protocols/rfc2616/rfc2616-sec5.html}{\textbf{\textit{żądania}}} HTTP (pola oddzielone spacjami):\\

\begin{code}
Method Request-URI HTTP-Version \r\n
HEADER1: VALUE1 \r\n
HEADER2: VALUE2 \r\n
...
HEADERX: VALUEX \r\n
\r\n
BODY
\r\n
\end{code}

\noindent gdzie:

\begin{itemize}\renewcommand{\labelitemii}{\scriptsize$\circ$}

\item \verb!Method! - to metoda żądania, dozwolone metody HTTP:
	\begin{itemize}
		\item \texttt{GET} – pobranie zasobu wskazanego przez \texttt{Request-URI}
		\item \texttt{HEAD} – pobiera informacje o zasobie, stosowane do sprawdzania dostępności zasobu
		\item \texttt{PUT} – przyjęcie danych przesyłanych od klienta do serwera, najczęściej aby zaktualizować wartość zasobu,
		\item \texttt{POST} – przyjęcie danych przesyłanych od klienta do serwera (np. wysyłanie zawartości formularzy),
		\item \texttt{DELETE} – żądanie usunięcia zasobu,
		\item \texttt{OPTIONS} – informacje o opcjach i wymaganiach dotyczących zasobu,
		\item \texttt{TRACE} – diagnostyka, analiza kanału komunikacyjnego,
		\item \texttt{CONNECT} – żądanie przeznaczone dla serwerów pośredniczących pełniących funkcje tunelowania,
		\item \texttt{PATCH} – aktualizacja części zasobu (np. jednego pola).
	\end{itemize}
	
\item \verb!Request-URI! - to ścieżka do zasobu na serwerze, która może zawierać dodatkowo parametry HTTP oraz fragment (za znakiem \#), 

\item \verb!HTTP-Version! - wersja protokołu HTTP, np. \texttt{HTTP/1.0}, \texttt{HTTP/1.1}, \texttt{HTTP/2.0}
\item \verb!HEADER1, HEADER2, ..., HEADERX! - nagłówki HTTP, \verb!VALLUE1, VALUE2, ..., VALUEX! - wartości konkretnych nagłówków
\item \verb!BODY! - opcjonalne ciało żądania
\end{itemize}
\end{itemize} 

\newpage 
\noindent {\underline{Odpowiedzi HTTP}} \\

\begin{itemize}
\renewcommand{\labelitemi}{\scriptsize}

\item Ogólny format \href{https://www.w3.org/Protocols/rfc2616/rfc2616-sec6.html}{\textbf{\textit{odpowiedzi}}} HTTP (pola oddzielone spacjami):\\

\begin{code}
HTTP-Version   Status-Code   Reason-Phrase   \r\n
HEADER1: VALUE1 \r\n
HEADER2: VALUE2 \r\n
...
HEADERX: VALUEX \r\n
\r\n
BODY
\r\n
\end{code}

\noindent gdzie:

\begin{itemize}\renewcommand{\labelitemii}{\scriptsize}
\item \verb!HTTP-Version! - wersja protokołu HTTP, np. \texttt{HTTP/1.0}, \texttt{HTTP/1.1}, \texttt{HTTP/2.0}
\item \verb!Status-Code! - \href{https://www.w3.org/Protocols/rfc2616/rfc2616-sec6.html#sec6.1.1}{\textbf{\textit{kod odpowiedzi}}}, który informuje klienta, w jaki sposób żądanie zostało lub nie zostało obsłużone, kody odpowiedzi to liczby trzycyfrowe, gdzie pierwsza z nich określa grupę odpowiedzi:

\begin{itemize}
\item \verb!1xx! - to kody informacyjne
\item \verb!2xx! - to kody powodzenia
\item \verb!3xx! - to kody przekierowania
\item \verb!4xx! - to kody błędu aplikacji klienta
\item \verb!5xx! - to kody błędu serwera
\end{itemize}

\item \verb!Reason-Phrase! - wiadomość powiązana z danym kodem odpowiedzi
\item \verb!HEADER1, HEADER2, ..., HEADERX! - nagłówki HTTP, \verb!VALLUE1, VALUE2, ..., VALUEX! - wartości konkretnych nagłówków
\item \verb!BODY! - opcjonalne ciało żądania
\end{itemize}

\end{itemize}\mbox{}

\noindent \underline{Dozwolone nagłówki HTTP}: \\

\begin{itemize} 
\item \href{https://tools.ietf.org/html/rfc2616#section-4.5}{\textbf{\textit{General Header Fields}}} -  are a few header fields which have general applicability for
   both request and response messages, but which do not apply to the
   entity being transferred.

\item \href{https://tools.ietf.org/html/rfc2616#section-7.1}{\textbf{\textit{Entity Header Fields}}} -   define metainformation about the entity-body or, if no body is present, about the resource identified by the request.

\item \href{https://tools.ietf.org/html/rfc2616#section-5.3}{\textbf{\textit{Request Header Fields}}} - allow the client to pass additional   information about the request, and about the client itself, to the server.

\item \href{https://tools.ietf.org/html/rfc2616#section-6.2}{\textbf{\textit{Response Header Fields}}} -  allow the server to pass additional
   information about the response which cannot be placed in the Status-
   Line. These header fields give information about the server and about
   further access to the resource identified by the Request-URI.

\end{itemize}\mbox{}


%\noindent \underline{Dozwolone nagłówki żądań HTTP} dostępne są pod adresem: \url{https://www.w3.org/Protocols/rfc2616/rfc2616-sec5.html#sec5.3}. \\

\newpage 
\noindent \underline{Przykłady żądań i odpowiedzi HTTP}: \\

\begin{itemize}\renewcommand{\labelitemii}{\scriptsize}
\item Żądanie:

\begin{code}
GET /index.html HTTP/1.1
HOST: 212.182.24.27


\end{code}

\item Odpowiedź:

\begin{code}
HTTP/1.1 200 OK
Date: Thu, 13 Apr 2017 14:25:38 GMT
Server: Apache/2.4.18 (Ubuntu)
Last-Modified: Thu, 13 Apr 2017 13:57:13 GMT
ETag: "2c39-54d0cb3af4405"
Accept-Ranges: bytes
Content-Length: 11321
Vary: Accept-Encoding
Content-Type: text/html


<!DOCTYPE html PUBLIC "-//W3C//DTD XHTML 1.0 Transitional//EN" 
"http://www.w3.org/TR/xhtml1/DTD/xhtml1-transitional.dtd">
<html xmlns="http://www.w3.org/1999/xhtml">
  <head>
    <meta http-equiv="Content-Type" content="text/html; charset=UTF-8" />
    <title>Apache2 Ubuntu Default Page: It works</title>
  </head>
  <body>
   ...
  </body>
</html>

\end{code}

\newpage 
\item Żądanie:

\begin{code}
TRACE / HTTP/1.1
HOST: 212.182.24.27

\end{code}

\item Odpowiedź:

\begin{code}
HTTP/1.1 405 Method Not Allowed
Date: Thu, 13 Apr 2017 14:31:22 GMT
Server: Apache/2.4.18 (Ubuntu)
Allow: 
Content-Length: 302
Content-Type: text/html; charset=iso-8859-1

<!DOCTYPE HTML PUBLIC "-//IETF//DTD HTML 2.0//EN">
<html><head>
<title>405 Method Not Allowed</title>
</head><body>
<h1>Method Not Allowed</h1>
<p>The requested method TRACE is not allowed for the URL /.</p>
<hr>
<address>Apache/2.4.18 (Ubuntu) Server at 212.182.24.27 Port 80</address>
</body></html>
\end{code}

\item Żądanie:

\begin{code}
OPTIONS /index.html HTTP/1.1
HOST: 212.182.24.27

\end{code}

\item Odpowiedź:

\begin{code}
HTTP/1.1 200 OK
Date: Thu, 13 Apr 2017 14:52:31 GMT
Server: Apache/2.4.18 (Ubuntu)
Allow: OPTIONS,GET,HEAD,POST
Content-Length: 0
Content-Type: text/html
\end{code}

\item Żądanie:

\begin{code}
HEAD /index.html HTTP/1.1
HOST: 212.182.24.27

\end{code}

\item Odpowiedź:

\begin{code}
HTTP/1.1 200 OK
Date: Thu, 13 Apr 2017 14:53:06 GMT
Server: Apache/2.4.18 (Ubuntu)
Last-Modified: Thu, 13 Apr 2017 13:57:13 GMT
ETag: "2c39-54d0cb3af4405"
Accept-Ranges: bytes
Content-Length: 11321
Vary: Accept-Encoding
Content-Type: text/html
\end{code}

\end{itemize}

